% !Mode:: "TeX:UTF-8"
% +-----------------------------------------------------------------------------
% | File: resume-zh
% | Author: luyunbo
% | E-mail: luyunbo47@gmail.com
% | Created: 2012-12-18
% | Last modified: 2013-03-16
% | Description:
% |     A Chinese Resume Example in LaTeX based on resumecls
% |
% | Copyrgiht (c) 2012-2013 by huxuan. All rights reserved.
% +-----------------------------------------------------------------------------

\documentclass[zh,color]{resumecls}
\name{卢云波}
\organization{6号楼406室}
\address{上海市长宁路865号,200050}
\mobile{+86 156 1862 2767}
\mail{luyunbo47@gmail.com}
\homepage{http://yblu.github.io}
\resumeurl{http://yblu.github.io/about}
\begin{document}
\begin{table}
\maketitle
%%%%%%%%%%%%%%%%%%%%%%%%%%%%%%%%%%%%%%%%%%%%%%%%%%%%%%%%%%%%%%%%%%%%%%%%%%%%%%%
\heading{教育经历}
\entry{2em}{Xrp{8em}}{
    \heiti{中国科学技术大学} & 合肥 & 2013/09至今 \\
}
\entry{4em}{lXX}{
    硕士 & 信息科学与技术学院 & 控制工程 \\
}
\entry{2em}{Xrp{8em}}{
    \heiti{中国科学技术大学} & 合肥 & 2009/08-2013/06 \\
}
\entry{4em}{lXX}{
    本科 & 信息科学与技术学院 & 自动化 \\   
}
%%%%%%%%%%%%%%%%%%%%%%%%%%%%%%%%%%%%%%%%%%%%%%%%%%%%%%%%%%%%%%%%%%%%%%%%%%%%%%%
%\heading{项目经历}
%\entry{2em}{Xp{8em}}{
%    \heiti{地点} & 起止时间 \\
%}
%\entry{4em}{X}{实验室名称 \quad 职位}
%\entry{6em}{X}{
%    研究方向和具体内容 \\
%    发表成果(亦可使用bibtex,像这样\cite{label},见文档最后注释内容) \\
%}
%%%%%%%%%%%%%%%%%%%%%%%%%%%%%%%%%%%%%%%%%%%%%%%%%%%%%%%%%%%%%%%%%%%%%%%%%%%%%%%
\heading{专业技能}
\entry{2em}{lX}{
    精通 & C/C++,Java,PHP,Matlab \\
    熟悉 & unix/linux,shell脚本 \\
    了解 & 分布式系统概念与架构 \\
    使用 & vim,git,svn,Linux(Ubuntu,Centos) \\
}
%%%%%%%%%%%%%%%%%%%%%%%%%%%%%%%%%%%%%%%%%%%%%%%%%%%%%%%%%%%%%%%%%%%%%%%%%%%%%%%
%\heading{工作经历}
%\entry{2em}{Xp{8em}}{
%    \heiti{单位名称} & 起止时间 \\
%}
%\entry{4em}{X}{部门 \quad 职位}
%\entry{6em}{X}{
%    负责的具体事项 \\
%    工作的具体内容 \\
%}
\heading{项目经历}
\entry{2em}{Xp{8em}}{
    \heiti{基于GPU+CPU的H.264解码器并行设计与实现} & 2013/09 - 2014/05 \\
}
%\entry{4em}{X}{部门 \quad 职位}
\entry{4em}{X}{
    研究H.264标准,阅读分析FFmpeg代码,抽取出基于FFmpeg的H.264解码器; \\
    测试了H264传统串行解码器的性能,分析模块并行化的可行性以及相应的并行粒度; \\
    设计H.264解码器的并行方案,其中包括解码器并行框架,数据结构与各功能模块接口;\\
    基于linux采用cuda编程实现一种基于CPU+GPU异构模型的H.264并行解码器,不仅大大减轻CPU的工作量于并提高解码速度。\\
}
\entry{2em}{Xp{8em}}{
    \heiti{基于web架构的智能家居系统开发} & 2012/08 \\
}
%\entry{4em}{X}{部门 \quad 职位}
\entry{4em}{X}{
    负责前端开发,通过整合Struts 2 框架、Hibernate框架及spring框架来开发智能家居管理系统。实现用户登录模块,用户注册模块,设备添加模块,删除设备模块以及更新设备信息模块; \\
     \\
}
\entry{2em}{Xp{8em}}{
    \heiti{独轮机器人的研究与实现} & 2011/06 - 2011/10 \\
}
%\entry{4em}{X}{部门 \quad 职位}
\entry{4em}{X}{
    team leader ,同时负责控制算法设计与实现。基于ARM处理器,利用陀螺仪,重力加速度传感器,光电编码器等进行信息采集,设计闭环控制算法,实现独轮机器人的站立与行走; \\
}
%%%%%%%%%%%%%%%%%%%%%%%%%%%%%%%%%%%%%%%%%%%%%%%%%%%%%%%%%%%%%%%%%%%%%%%%%%%%%%%
%\heading{校园经历}
%\entry{2em}{Xp{8em}}{
%    经历1 & 起止时间 \\
%    经历2 & 起止时间 \\
%}
%%%%%%%%%%%%%%%%%%%%%%%%%%%%%%%%%%%%%%%%%%%%%%%%%%%%%%%%%%%%%%%%%%%%%%%%%%%%%%%
\heading{获得荣誉}
\entry{2em}{Xr}{
    2012年获第十四届“挑战杯”全国大学生课外学术科技作品校内选拨赛一等奖 \\
    2011年中国科学技术大学RoboGame机器人比赛杂技组冠军 \\
    2009~2010,20010~2011, 2011~2012学年均获校优秀学生奖学金铜奖  \\
    2011年被评为优秀学生干部 \\
}
%%%%%%%%%%%%%%%%%%%%%%%%%%%%%%%%%%%%%%%%%%%%%%%%%%%%%%%%%%%%%%%%%%%%%%%%%%%%%%%
%\heading{其他列举事项-如个人爱好,网络资料等}
%\entry{2em}{lX}{
%    标签1 & 标签对应内容 \\
%    标签2 & 标签对应内容 \\
%}
%%%%%%%%%%%%%%%%%%%%%%%%%%%%%%%%%%%%%%%%%%%%%%%%%%%%%%%%%%%%%%%%%%%%%%%%%%%%%%%
% 如果不需要发表成果,注释这一段即可
\heading{科研成果}
\vspace{-6em}
\bibliography{resume}
%%%%%%%%%%%%%%%%%%%%%%%%%%%%%%%%%%%%%%%%%%%%%%%%%%%%%%%%%%%%%%%%%%%%%%%%%%%%%%%
\end{table}
\end{document}
